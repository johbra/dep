% Created 2019-08-28 Wed 14:14
% Intended LaTeX compiler: pdflatex
\documentclass[11pt]{article}
\usepackage[utf8]{inputenc}
\usepackage[T1]{fontenc}
\usepackage{graphicx}
\usepackage{grffile}
\usepackage{longtable}
\usepackage{wrapfig}
\usepackage{rotating}
\usepackage[normalem]{ulem}
\usepackage{amsmath}
\usepackage{textcomp}
\usepackage{amssymb}
\usepackage{capt-of}
\usepackage{hyperref}
\author{Johannes Brauer}
\date{\today}
\title{Portierung der Dozenteneinsatzplanung\\\medskip
\large Wozu brauchen wir OOP? Diskussion am 18.12.18}
\hypersetup{
 pdfauthor={Johannes Brauer},
 pdftitle={Portierung der Dozenteneinsatzplanung},
 pdfkeywords={},
 pdfsubject={},
 pdfcreator={Emacs 25.3.50.1 (Org mode 9.2.5)}, 
 pdflang={English}}
\begin{document}

\maketitle
\section*{Warum?}
\label{sec:org979a519}
Im Rahmen der Diskussionsrunde \emph{Wozu brauchen wir OOP?}, die auf dem
Pamphlet mit dem Titel \href{https://www.dropbox.com/s/36vsbdk1bn4w17o/oop-wozu.pdf?dl=0}{Brauchen wir objektorientierte Programmierung?}
basierte, wurde vereinbart, dass der Autor versucht, die Erfahrungen, die in
dem genannten Pamphlet lediglich auf der Erstellung einer funktionalen
Version für \emph{NoNopoly} (einer stark vereinfachten Monopoly-Simulation)
basierte, dadurch zu vertiefen, dass eine kleine -- aber
nicht-triviale -- Anwendung von einer objektorientierten in
eine funktionale Version portiert werden sollte.

Als Untersuchungsgegenstand wurde die seit mehr als einem Jahrzehnt an
der NAK benutzte Dozenteneinsatzplanung gewählt. Dabei sollte
insbesondere auch die Erstellung einer Web-Oberfläche einbezogen
werden, auf die bei der NoNopoly-Portierung verzichtet wurde.
\section*{Ausgangspunkt}
\label{sec:orge7ee835}
Die Dozenteneinsatzplanung in der heute im Fachbereich benutzen Form
ist um 2004 in einer ersten Fassung mit VisualWorks-Smalltalk
entstanden. Sie ist nach dem Model-View-Controller-Prinzip entwickelt
worden und war zu Beginn mit einer nativen VisualWorks-Oberfläche
ausgestattet. Diese wurde später durch eine Weboberfläche unter
Verwendung des Frameworks Seaside (Smalltalk) ersetzt. Aufgrund
der konsequenten Nutzung von MVC konnte dieser Vorgang ohne Änderung
an den Model-Klassen vollzogen werden.

Der letzte größere Entwicklungsschritt war dann die Portierung nach
Pharo-Smalltalk (open source).
\subsubsection*{Eckdaten der Smalltalk-Version}
\label{sec:orgb27fc60}
\begin{itemize}
\item 8 Model-Klassen mit ca. 220 Methoden; ca. 1500 LoC
\item keine Hierarchie unter den Model-Klassen
\item 19 Seaside-Klassen (Komponenten) mit 199 Methoden; ca. 2600 LoC
\end{itemize}
\section*{Ziel}
\label{sec:org413ab4c}
Wesentliche Komponenten der Dozenteneinsatzplanung sollten von
Smalltalk (Pharo) nach Clojure/ClojureScript (s.~u.) portiert werden.
\section*{Zur Wahl der Zielsprache}
\label{sec:org849b6e5}
Die Begründung, als funktionale Sprache Clojure(script) zu verwenden,
ist hauptsächlich in den Vorkenntnisse des Autors zu sehen. Ein
irgendwie geartetes Auswahlverfahren hat nicht stattgefunden, wenn man
von einigen Versuchen mit der Sprache \emph{\href{https://elm-lang.org}{Elm}} absieht.
\subsection*{Exkurs: Stichworte zu den Versuchen mit \href{https://elm-lang.org}{Elm}}
\label{sec:org8d95680}
\begin{itemize}
\item Eigenschaften:
\begin{itemize}
\item dient der Entwicklung von Webanwendungen
\item strenge, statische Typisierung
\item übersetzt nach Javascript
\item „Fazit”: Wenn man den Type-Checker zufrieden gestellt hat,
funktioniert die Anwendung,
\item „Nachteil”: „Quick-and-dirty“-Programmierung im Sinne von „mal eben
etwas ausprobieren“ ist nicht möglich.
\item Elm kennt keine Komponenten (im Sinne von \href{https://reactjs.org}{React}, s.~u.)
\item Diskussion über die Architektur größerer Elm-Anwendungen im vollen Gange
\item Elm noch nicht sehr stabil
\item Ein Minibeispiel für eine Dozenten-Pflege existiert: \href{./Dozent-V0.elm}{Quelltext}
\begin{verbatim}
    cd /Users/jb/Dropbox/elm/dep 
    elm reactor
    ... Go to <http://localhost:8000> to see your project dashboard.
\end{verbatim}
\end{itemize}
\end{itemize}
\subsection*{Eckdaten der Clojure/ClojureScript-Version}
\label{sec:orge31df31}
\subsubsection*{Vorbemerkung zu Clojure vs. ClojureScript}
\label{sec:org27de1e0}
\href{https://clojure.org}{Clojure} ist ein Lisp-Dialekt, der auf der JVM läuft. \href{https://clojurescript.org}{ClojureScript}
hingegen ist ein Compiler für Clojure, der nach JavaScript übersetzt.
Die erste Idee für die Portierung der Dozenteneinsatzplanung bestand
in der Entwicklung einer Client-Server-Anwendung. Dabei sollten die
Model-Klassen nach Clojure portiert und dann nach einer
geeigneten Variante für die Realisierung der Weboberfläche Ausschau
gehalten werden.

Der erste Teil hat stattgefunden. Im Zuge der Überlegungen zur
Weboberfläche hat sich der Autor dann entschieden, statt einer
Client-Server-Anwendung eine Single-Page-Application zu bauen und
dafür dann ClojureScript zu verwenden. Die Anwendung läuft also
vollständig im Web-Browser ab. Die schon entstandenen Clojure-Dateien
für den Model-Part konnten dabei im Wesentlichen unverändert übernommen
werden. (Die Dateiendungen müssen von .clj in .cljc geändert werden.)

\subsubsection*{Web-Technologie}
\label{sec:orgc6af341}
Für die Entwicklung der Single-Page-Application lag es nun nahe, diese
nach dem Konzept des \emph{\href{https://en.wikipedia.org/wiki/Functional\_reactive\_programming}{Functional Reactive Programming}} zu bauen. Für
JavaScript gibt es dafür verschiedene Bibliotheken (z.~B. \href{https://reactjs.org}{React},
\href{https://angular.io}{Angular}, oder \href{https://vuejs.org}{Vue.js}). Das entscheidende Auswahlkriterium für den
Autor, der zuvor noch nie eine SPA entwickelt und von JavaScript keine
Ahnung hat, war, eine ClojureScript-Bibliothek zu finden, mit deren Hilfe
FRP möglich ist, ohne in JavaScript programmieren zu müssen. So ist
die Entscheidung für \href{https://github.com/reagent-project/reagent}{Reagent} (A minimalistic ClojureScript interface
to React.js) zustande gekommen.

Vereinfacht gesprochen besteht das Grundprinzip von Reagent darin, in
einer (oder mehreren) ClojureScript-Variablen den Weltzustand zu
speichern. Sollte sich durch Benutzer-Interkation dieser Zustand
ändern, sorgt Reagent automatisch für das Rerendering der zugehörigen
React-Komponenten im Browser. 

\subsubsection*{Zahlen}
\label{sec:org9946ac7}
Bei den in der Folge genannten Definitionen handelt es überwiegend um
Funktionen ergänzt um ein paar Variablendefinitionen.
\begin{itemize}
\item Model: 5 cljc-Dateien mit 71 Definitionen; ca. 550 LoC
\item Weboberfläche (cljs-Dateien):
\begin{itemize}
\item core.cljs - 4 Definitionen; 45 LoC
\item state.cljs - 4 Definitionen; 40 LoC
\item 5 Reagent-Komponenten mit 21 Definitionen; ca. 360 LoC
\item 12 Hilfsfunktionen; ca. 230 LoC
\end{itemize}
\end{itemize}
\subsection*{Vergleich der Eckdaten}
\label{sec:org29abe02}

\begin{center}
\begin{tabular}{lrrrr}
 & Smalltalk &  & ClojureScript & \\
\hline
 & Anz. Methoden & LoC & Anz. Definitionen & LoC\\
\hline
Model & 220 & 1500 & 71 & 550\\
Oberfläche & 199 & 2600 & 47 & 675\\
\hline
Summe LoC &  & 4100 &  & 1225\\
\hline
\end{tabular}
\end{center}

Selbst wenn man berücksichtigt, dass
\begin{itemize}
\item der ClojureScript-Variante noch ein paar Kleinigkeiten fehlen (Der
aktuelle Prototyp kann unter \url{https://johbra.github.io/dep/} besichtigt
werden), so z.~B. 
\begin{itemize}
\item eine Auswahlbox für Dozenten in Lehrveranstaltungen,
\item das Anlegen eines neuen Geschäftsjahres,
\item ein paar Sicherheitsabfragen z.~B. beim Löschen von Moduln
oder Manipeln oder der Neuplanung eines Quartals sowie
\item der Änderung der Semesterzuordnung für Manipel
\end{itemize}
\item ein Redesign der Smalltalk-Variante hier und dort Einsparungen
erbringen könnte,
\end{itemize}
dürfte die Smalltalk-Variante sicherlich um den Faktor 2 bezogen auf
die LoC größer bleiben.

\section*{Erfahrungen mit Clojure/ClojureScript}
\label{sec:orgf35563c}
Wie bereits in \href{https://www.dropbox.com/s/36vsbdk1bn4w17o/oop-wozu.pdf?dl=0}{Brauchen wir objektorientierte Programmierung?}
ausgeführt, lautet die Standard-Antwort von Clojure-Entwicklern
bezüglich der Frage „Was machen wir mit den Objekten/Klassen?“: Man
nehme Hashmaps. Insbesondere im Abschnitt \emph{NoNopoly mit Hashmaps} ist
dargelegt worden, dass eine strikt funktionale Implementierung nicht
nur möglich, sondern in einer funktionalen Sprache wohl auch die
angemessene Vorgehensweise ist. Die Verwendung einer DSL, die die
Benutzung von Klassen und Objekten im Sinne der gängigen
objektorientierten Sichtweise ermöglicht, hat keinen wirklichen Nutzen
erkennen lassen. Auch die Verwendung von \href{https://clojure.org/reference/datatypes}{Clojure-Datentypen} und
\href{https://clojure.org/reference/protocols}{Protokollen} scheint nur dann sinnvoll zu sein, wenn für die zu
erstellende Anwendung die Java-Interoperabilität von besonderer
Bedeutung ist.

Der Beschränkung auf die grundständigen Clojure-Datenstrukturen (Neben
den Hashmaps ist noch von Vektoren Gebrauch gemacht) worden, hat auch bei
der Implementierung der Dozenteneinsatzplanung keinerlei
Schwierigkeiten bereitet. Im Gegenteil bietet die Vorgehensweise
einige Vorteile:

\begin{itemize}
\item Man spart das Anlegen von Klassen sowie von Getters und Setters.
\item Man kann jederzeit aus Einzelheiten eine Struktur (Hashmap)
erstellen, um sie z.~B. als Argument an eine Funktion zu
übergeben. Man erzeugt quasi „Ad-hoc“-Objekte. Dass nicht zuvor eine
Klasse erstellt werden muss, ist überaus nützlich.
\item Gerade bei der objektorientierten Programmierung von Oberflächen hat man
es überwiegend mit Singletons zu tun. Keine Programmiersprache
sollte einen Programmierer dazu zwingen, erst eine Klassen anlegen
zu müssen, nur um davon anschließend ein einziges Exemplar zu erzeugen.
\end{itemize}

Eines der zentralen Anliegen der Objektorientierung besteht in der
Vereinigung von Struktur und Verhalten. Nach den nunmehrigen
Erfahrungen des Autors wird dadurch Wiederverwendung stark
eingeschränkt. Es ist selten sinnvoll, dass eine Funktion/Methode nur
im Kontext eines Objekts (bzw. einer Klasse) verwendet werden
darf. Dies führt eher zu einer „Explosion“ der Anzahl der
Methoden. Das Zitat von Alan Perlis: „It is better to have 100
functions operate on one data structure than 10 functions on 10 data
structures.“ hat sich bei der Portierung der Dozenteneinsatzplanung bestätigt:
\begin{itemize}
\item Es wurden kaum eigene Funktionen entwickelt.
\item Es gibt fast immer eine Standard- oder Bibliotheksfunktion, die das
gegebene Problem löst.
\item Dabei ist die intensive Verwendung der klassischen Funktionen
höherer Ordnung (map, filter, reduce) überaus hilfreich.
\end{itemize}
Aber auch selbst definierte Funktionen höherer Ordnung führen zu kompakten
Lösungen. Dies ist in einer Funktion, die alle modalen Fenster für die
unterschiedlichen Pflege-Dialoge (für Dozenten, Manipel, Module und
Lehrveranstaltungen) erstellt, „auf die Spitze getrieben worden“, was
dazu führte, dass die Funktion zwar kompakt aber auch sehr komplex
ist.
\subsection*{Objektgeflechte}
\label{sec:org009efe7}
Bei der Entwicklung der Smalltalk-Variante schien es dem Autor
naheliegend, immer wieder mit Objektgeflechten zu arbeiten. So hat ein
Lehrveranstaltungs-Objekt je einen Verweis auf ein Dozenten-, ein
Modul- und ein Manipel-Objekt. Durch Verfolgen der Verweise kann
zwischen den Objekte programmtechnisch bequem navigiert werden.

Allerdings hat dies zur Folge, dass dem Anwender nicht so ohne
weiteres gestattet werden kann, z.~B. ein nach aktuellen
Studienplänen nicht mehr gültiges Modul zu löschen, solange des noch
Lehrveranstaltungs-Objekte gibt, die auf das Modul verweisen. Der
Benutzer bekommt daher den Hinweis, dass das Modul nicht gelöscht werden
kann, solange es noch Lehrveranstaltungen gibt, die den Modul-Verweis
besitzen. Andernfalls drohte ein Crash der Anwendung.

In der funktionalen Variante gibt es keinerlei Objektverweise. In der
Lehrveranstaltungs-Hashmap steht der Name des Dozenten, die
Modulnummer und die Manipelkennung. Sollte das Modul mit der Nummer
gelöscht werden, ist die „Lebensfähigkeit“ der
Lehrveranstaltungs-Hashmap und der Anwendung insgesamt nicht
bedroht. Gleichwohl könnte es im Sinne der Benutzungsfreundlichkeit
sinnvoll sein, einen entsprechenden Warnhinweis zu geben, was zur
Zeit -- wie oben erwähnt -- bisher nicht implementiert ist.

Selbstverständlich könnte man in der objektorientierten Variante auch
auf direkte Objektverweise verzichten. Aber wäre das nicht ein
weiteres Argument dafür, auf Objekte zu verzichten?

\subsection*{Web-Oberfläche}
\label{sec:orgd8a178f}
Bei der Erstellung der Web-Oberfläche unter Verwendung der Bibliothek
Reagent kann natürlich nicht im engeren Sinne von einer Portierung der
Smalltalk-Seaside-Oberfläche gesprochen werden. Dafür sind die
verwendeten Technologien zu unterschiedlich. Die
Smalltalk-Variante ist als Client-Server-Anwendung realisiert, bei der
auf die Verwendung von JavaScript-Elementen fast vollständig
verzichtet wurde, während die ClojureScript-Version vollständig
im Browser läuft. Selbst wenn die funktionale Version auch als
Client-Server-Anwendung implementiert worden wäre, wären Bibliotheken
zum Einsatz gekommen, die keinerlei Verwandtschaft mit dem
Seaside-Framework aufweisen. 

Eine Gemeinsamkeit gibt es dennoch. Typisch für Seaside ist, dass der
Programmierer keinen HTML-Code schreibt, sondern stattdessen die von
Seaside bereitgestellten Nachrichten benutzt, aus denen dann HTML-Code
generiert wird. Eine ähnliche Vorgehensweise ist auch in
Clojure/ClojureScript üblich. In Reagent aber auch in anderen
Web-Bibliotheken wird die Bibliothek \href{https://github.com/weavejester/hiccup}{Hiccup} verwendet, die erlaubt,
HTML in Clojure-Syntax aufzuschreiben. HTML-Tags werden als Vektoren
und Attribute als Hashmaps aufgeschrieben.

\section*{Vorläufiges Fazit (sehr vorläufig)}
\label{sec:org779759d}
Der Autor hat sich im Verlauf der Arbeiten an der Portierung gefragt,
welches der die Objektorientierung kennzeichnenden Prinzipien
(Klassenprinzip, Kapselung, Polymorhie und Vererbung) er vielleicht in
der funktionalen Welt vermisst hat.
\begin{quote}
Nebenbei bemerkt hat die Objektorientierung der Welt der
Programmiersprachen lediglich die Vererbung als Neuerung
hinzugefügt. Alles andere gab es auch schon vorher. Vererbung gilt
gleichzeitig als hoch problematisch. Dies hat viel damit zu tun, dass
in der Objektorientierung nie geklärt wurde, welchem Zweck
Klassenhierarchien eigentlich dienen sollen.
\end{quote}
Um es vorweg zu nehmen, vermisst wurde nichts. Die Aussage bezieht
sich aber nur auf das hier betrachtete Beispiel der
Dozenteneinsatzplanung. Dabei handelt es sich ja um eine
administrative Anwendung mit einer überschaubaren Anwendungslogik. Es
ist durchaus denkbar, dass in einem anderen Kontext -- z.~B. in
einer Simulation, bei der man vielleicht reale Objekte im Rechner
rekonstruieren möchte -- andere Erkenntnisse gewonnen werden könnten.
\subsection*{Kapselung}
\label{sec:org9bf0d36}
Kapselung von Strukturdetails von Hashmaps ist in der funktionalen
Programmierung generell und auch in Clojure/ClojureScript kein
Thema. Begrenzungen der Sichtbarkeit von Definitionen sind an
Modulgrenzen möglich. Davon wurde in der Dozenteneinsatzplanung aber
kein Gebrauch gemacht.

\subsection*{Klassen, Vererbung}
\label{sec:orga26c071}
Wie oben bereits erläutert, sind Klassen als Baupläne für Objekte
vollkommen entbehrlich. Zu der Frage, ob Klassen als Elemente von
Hierarchien nützlich sind, liefert die Portierung keine Anhaltspunkte,
da unter den Model-Klassen von Hierarchien kein Gebrauch gemacht
wurde.

Ob das Klassenprinzip überhaupt ein wesentlicher Bestandteil der
Objektorientierung ist, kann durchaus infrage gestellt
werden. Schließlich gibt es auch klassenlose objektorientierte
Programmiersprachen. 

Klassenhierarchien, Vererbung und Unterklassen-Polymorphie gehören
jedenfalls nicht dazu, wenn man die Definition des Terminus
„objektorientiert“ betrachtet, die Alan Kay, dem Erfinder des Begriffs,
gegeben hat:
\begin{quote}
„OOP to me means only messaging, local retention and protection and
hiding of state-process, and extreme late-binding of all things.“ 
\textasciitilde{}Alan Kay
\end{quote}
Mit anderen Worten, gemäß Alan Kay sind die wesentlichen Bestandteile
von OOP:
\begin{itemize}
\item Message passing
\item Encapsulation
\item Dynamic binding
\end{itemize}

\subsection*{Polymorphie}
\label{sec:org1a482f0}
Die gängigen objektorientierten Sprachen kennen dynamische Bindung nur
in der primitiven Form der Unterklassen-Polymorphie. Da in den
Model-Klassen der Smalltalk-Variante keine Klassenhierarchien
existieren, spielt darauf aufbauende Polymorphie keine Rolle. Insofern
trat diesbezüglich der Portierung auch kein Problem auf.

Es soll an dieser Stelle nicht unerwähnt bleiben, dass
Clojure/ClojureScript mit dem Konzept der \href{https://clojure.org/reference/multimethods}{Multimethods} über ein
mächtiges Werkzeug für späte Bindung verfügt. Während in Smalltalk
oder Java bei einem Ausdruck der Form \texttt{a.f(b)} die Auswahl der Methode
\texttt{f} (absurderweise) ausschließlich von der Klassenzugehörigkeit von
\texttt{a} abhängt (die von \texttt{b} bleibt unberücksichtigt), würde in Clojure
eine Multimethod \texttt{f} mit einer Dispatcher-Funktion ausgestattet, über
die der Bindungsmechanismus nach Belieben ausgestaltet werden kann.

\subsection*{Entwicklungswerkzeuge}
\label{sec:orga40a768}

Hier gibt es eigentlich nichts Bedeutendes zu berichten. Aber wenn
auch die Arbeit mit der REPL einen dynamischen und produktiven
Entwicklungsprozess erlaubt, so ist dem
Autor zum wiederholten Male deutlich geworden, dass die
Leistungsfähigkeit einer Smalltalk-Entwicklungsumgebung, wie sie von
VisualWorks oder Pharo bereitgestellt wird, unerreicht ist.
\section*{Zum Schluss \ldots{}}
\label{sec:orgbcf11e0}
\ldots{} noch zwei „Zitate”: 
\begin{itemize}
\item \begin{center}
\includegraphics[width=.5\linewidth]{./problemfactory.jpeg}
\end{center}
\item Object oriented programs are offered as alternatives to correct ones…

Edsger W. Dijkstra, pioneer of computer science
\end{itemize}
\end{document}